% vim: set textwidth=120:

% Example CV based on the 1.5-column-cv template. Main features:
% * uses the Roboto font family and IcoMoon icon set;
% * doesn't use colours, different font weights are used instead for styling;
% * because the CV fits on one page, header and footer is empty, since there isn't much useful info to put there;
% * includes a photo.
\documentclass[a4paper,10pt]{article}


% package imports
% ---------------

\usepackage[british]{babel} % for correct language and hyphenation and stuff
\usepackage{calc}           % for easier length calculations (infix notation)
\usepackage{enumitem}       % for configuring list environments
\usepackage{fancyhdr}       % for setting header and footer
\usepackage{fontspec}       % for fonts
\usepackage{geometry}       % for setting margins (\newgeometry)
\usepackage{graphicx}       % for pictures
\usepackage{microtype}      % for microtypography stuff
\usepackage{xcolor}         % for colours


% margin and column widths
% ------------------------

% margins
\newgeometry{left=15mm,right=15mm,top=15mm,bottom=15mm}

% width of the gap between left and right column
\newlength{\cvcolumngapwidth}
\setlength{\cvcolumngapwidth}{3.5mm}

% left column width
\newlength{\cvleftcolumnwidth}
\setlength{\cvleftcolumnwidth}{44.5mm}

% right column width
\newlength{\cvrightcolumnwidth}
\setlength{\cvrightcolumnwidth}{\textwidth-\cvleftcolumnwidth-\cvcolumngapwidth}

% set paragraph indentation to 0, because it screws up the whole layout otherwise
\setlength{\parindent}{0mm}


% style definitions
% -----------------
% style categories explanation:
% * \cvnameXXX is used for the name;
% * \cvsectionXXX is used for section names (left column, accompanied by a horizontal rule);
% * \cvtitleXXX is used for job/education titles (right column);
% * \cvdurationXXX is used for job/education durations (left column);
% * \cvheadingXXX is used for headings (left column);
% * \cvmainXXX (and \setmainfont) is used for main text;
% * \cvruleXXX is used for the horizontal rules denoting sections.

% font families
\defaultfontfeatures{Ligatures=TeX} % reportedly a good idea, see https://tex.stackexchange.com/a/37251

\newfontfamily{\cvnamefont}{Roboto Medium}
\newfontfamily{\cvsectionfont}{Roboto Medium}
\newfontfamily{\cvtitlefont}{Roboto Regular}
\newfontfamily{\cvdurationfont}{Roboto Light Italic}
\newfontfamily{\cvheadingfont}{Roboto Regular}
\setmainfont{Roboto Light}

% colours
\definecolor{cvnamecolor}{HTML}{000000}
\definecolor{cvsectioncolor}{HTML}{000000}
\definecolor{cvtitlecolor}{HTML}{000000}
\definecolor{cvdurationcolor}{HTML}{000000}
\definecolor{cvheadingcolor}{HTML}{000000}
\definecolor{cvmaincolor}{HTML}{000000}
\definecolor{cvrulecolor}{HTML}{000000}

\color{cvmaincolor}

% styles
\newcommand{\cvnamestyle}[1]{{\Large\cvnamefont\textcolor{cvnamecolor}{#1}}}
\newcommand{\cvsectionstyle}[1]{{\normalsize\cvsectionfont\textcolor{cvsectioncolor}{#1}}}
\newcommand{\cvtitlestyle}[1]{{\large\cvtitlefont\textcolor{cvtitlecolor}{#1}}}
\newcommand{\cvdurationstyle}[1]{{\small\cvdurationfont\textcolor{cvdurationcolor}{#1}}}
\newcommand{\cvheadingstyle}[1]{{\normalsize\cvheadingfont\textcolor{cvheadingcolor}{#1}}}


% inter-item spacing
% ------------------

% vertical space after personal info and standard CV items
\newlength{\cvafteritemskipamount}
\setlength{\cvafteritemskipamount}{5mm plus 1.25mm minus 1.25mm}

% vertical space after sections
\newlength{\cvaftersectionskipamount}
\setlength{\cvaftersectionskipamount}{2mm plus 0.5mm minus 0.5mm}

% extra vertical space to be used when a section starts with an item with a heading (e.g. in the skills section),
% so that the heading does not follow the section name too closely
\newlength{\cvbetweensectionandheadingextraskipamount}
\setlength{\cvbetweensectionandheadingextraskipamount}{1mm plus 0.25mm minus 0.25mm}


% intra-item spacing
% ------------------

% vertical space after name
\newlength{\cvafternameskipamount}
\setlength{\cvafternameskipamount}{3mm plus 0.75mm minus 0.75mm}

% vertical space after personal info lines
\newlength{\cvafterpersonalinfolineskipamount}
\setlength{\cvafterpersonalinfolineskipamount}{2mm plus 0.5mm minus 0.5mm}

% vertical space after titles
\newlength{\cvaftertitleskipamount}
\setlength{\cvaftertitleskipamount}{1mm plus 0.25mm minus 0.25mm}

% value to be used as parskip in right column of CV items and itemsep in lists (same for both, for consistency)
\newlength{\cvparskip}
\setlength{\cvparskip}{0.5mm plus 0.125mm minus 0.125mm}

% set global list configuration (use parskip as itemsep, and no separation otherwise)
\setlist{parsep=0mm,topsep=0mm,partopsep=0mm,itemsep=\cvparskip}


% CV commands
% -----------

% creates a "personal info" CV item with the given left and right column contents, with appropriate vertical space after
% @param #1 left column content (should be the CV photo)
% @param #2 right column content (should be the name and personal info)
\newcommand{\cvpersonalinfo}[2]{
    % left and right column
    \begin{minipage}[t]{\cvleftcolumnwidth}
        \vspace{0mm} % XXX hack to align to top, see https://tex.stackexchange.com/a/11632
        \raggedleft #1
    \end{minipage}% XXX necessary comment to avoid unwanted space
    \hspace{\cvcolumngapwidth}% XXX necessary comment to avoid unwanted space
    \begin{minipage}[t]{\cvrightcolumnwidth}
        \vspace{0mm} % XXX hack to align to top, see https://tex.stackexchange.com/a/11632
        #2
    \end{minipage}

    % space after
    \vspace{\cvafteritemskipamount}
}

% typesets a name, with appropriate vertical space after
% @param #1 name text
\newcommand{\cvname}[1]{
    % name
    \cvnamestyle{#1}

    % space after
    \vspace{\cvafternameskipamount}
}

% typesets a line of personal info beginning with an icon, with appropriate vertical space after
% @param #1 parameters for the \includegraphics command used to include the icon
% @param #2 icon filename
% @param #3 line text
\newcommand{\cvpersonalinfolinewithicon}[3]{
    % icon, vertically aligned with text (see https://tex.stackexchange.com/a/129463)
    \raisebox{.5\fontcharht\font`E-.5\height}{\includegraphics[#1]{#2}}
    % text
    #3

    % space after
    \vspace{\cvafterpersonalinfolineskipamount}
}

% creates a "section" CV item with the given left column content, a horizontal rule in the right column, and with
% appropriate vertical space after
% @param #1 left column content (should be the section name)
\newcommand{\cvsection}[1]{
    % left and right column
    \begin{minipage}[t]{\cvleftcolumnwidth}
        \raggedleft\cvsectionstyle{#1}
    \end{minipage}% XXX necessary comment to avoid unwanted space
    \hspace{\cvcolumngapwidth}% XXX necessary comment to avoid unwanted space
    \begin{minipage}[t]{\cvrightcolumnwidth}
        \textcolor{cvrulecolor}{\rule{\cvrightcolumnwidth}{0.3mm}}
    \end{minipage}

    % space after
    \vspace{\cvaftersectionskipamount}
}

% creates a standard, multi-purpose CV item with the given left and right column contents, parskip set to cvparskip
% in the right column, and with appropriate vertical space after
% @param #1 left column content
% @param #2 right column content
\newcommand{\cvitem}[2]{
    % left and right column
    \begin{minipage}[t]{\cvleftcolumnwidth}
        \raggedleft #1
    \end{minipage}% XXX necessary comment to avoid unwanted space
    \hspace{\cvcolumngapwidth}% XXX necessary comment to avoid unwanted space
    \begin{minipage}[t]{\cvrightcolumnwidth}
        \setlength{\parskip}{\cvparskip} #2
    \end{minipage}

    % space after
    \vspace{\cvafteritemskipamount}
}

% typesets a title, with appropriate vertical space after
% @param #1 title text
\newcommand{\cvtitle}[1]{
    % title
    \cvtitlestyle{#1}

    % space after
    \vspace{\cvaftertitleskipamount}
    % XXX need to subtract cvparskip here, because it is automatically inserted after the title "paragraph"
    \vspace{-\cvparskip}
}


% header and footer
% -----------------

% set empty header and footer
\pagestyle{empty}



% preamble end/document start
% ===========================

\begin{document}


% personal info
% -------------

\cvpersonalinfo{
    % photo
    \includegraphics[height=36mm]{photo.jpg}
}{
    % name
    \cvname{Mario Alberto Rosales}

    % address
    \cvpersonalinfolinewithicon{height=4mm}{072-location.pdf}{
        Barrio El Pozo - Manz. 12 Viv. 19, Santa Fe
    }

    % phone number
    \cvpersonalinfolinewithicon{height=4mm}{067-phone.pdf}{
        +549 0342\,155\, 947 \,097
    }

    % email address
    \cvpersonalinfolinewithicon{height=4mm}{070-envelop.pdf}{
        mariorosales941@gmail.com.com
    }

    % LinkedIn account
    \cvpersonalinfolinewithicon{height=4mm}{458-linkedin.pdf}{
        mario-rosales
    }

    % date of birth
    2 de Enero de 1991
}


% work experience
% ---------------

\cvsection{EXPERIENCIA LABORAL}

% ANCA
\cvitem{
    \cvdurationstyle{May 2017 - Actualidad}
}{
    \cvtitle{Contrato}
    Desarrollador Web en ANCA SRL, Santa Fe.
}

% Sadosky
\cvitem{
    \cvdurationstyle{Jul 2015 - Ago 2016}
}{
    \cvtitle{Instructor en Escuelas Secundarias, Fundación Manuel Sadosky - Universidad Nacional del Litoral, Santa Fe.}
    Beca de trabajo para estudiantes en Servicio a Tercero destinado a dictar cursos de programación básica en escuelas secundarias de la ciudad de Santa Fe, y promocionar actividades de la Facultad de Ingeniería y Ciencias Hídricas y de la Fundación Dr. Manuel Sadosky de Investigación y Desarrollo en las Tecnologías de la Información y la Comunicación.
}

% Pasantia no Resntada
\cvitem{
    \cvdurationstyle{Jul-Dic 2009}
}{
    \cvtitle{Pasantía no Rentada}
    Dirección Provincial de Catastro - Santa Fe
}

% education
% ---------

\cvsection{FORMACIÓN ACADÉMICA}

% FICH
\cvitem{
    \cvdurationstyle{2010 - Actualidad}
}{
    \cvtitle{Estudios Universitarios}
    Ingeniería en Informática, Facultad de Ingeniería y Ciencias Hídricas - Universidad Nacional del Litoral, Santa Fe. Actualmente desarrollando el Proyecto Final de Carrera.
    \begin{itemize}
        \item Tesis: Diseño y Desarrollo de un Software de Gestión y Seguimiento de Trazabilidad de Ganado Vacuno.
        \begin{itemize}
            \item Objetivo: Diseñar y desarrollar un sistema web destinado a los criadores y productores de ganado vacuno con el fin
de brindarles una herramienta para la gestión y seguimiento de trazabilidad de bovinos.
        \end{itemize}
    \end{itemize}
}

% Inglés Almirante Brown
\cvitem{
    \cvdurationstyle{2003-2009}
}{
    \cvtitle{Idioma Extrangero}
    Instituto Superior de Profesorado No 8 – ALMIRANTE GUILLERMO BROWN.
}

%Secundaria
\cvitem{
    \cvdurationstyle{2003-2009}
}{
    \cvtitle{Estudios Secundarios}
    Escuela de Enseñansa Técnica Nro. 480 Manuel Belgrano, Santa Fe. Modalidad Técnico en Informática
}

% Cursos y Certificados
\cvsection{CURSOS Y CERTIFICADOS}

\cvitem{
    \cvheadingstyle{Certificado de Aprobación}
}{
    \cvtitle{Curso de Desarrollo de Aplicaciones Web} Nivel 1 y 2, Facultad de Ingeniería y Ciencias Hídricas - UNL, Santa Fe.
    \begin{itemize}
        \item Nivel 1: Duración 30hs (Los días 13, 20 y 27 de Octubre, y 3, 10 y 17 de Noviembre de 2012) - Evaluación final: SI.
        \item Nivel 2: Duración 30hs (Los días 4, 11 y 18 de Mayo, y 1, 8 y 15 de Junio de 2013) - Evaluación final: SI.
    \end{itemize}
}

\cvitem{
    \cvheadingstyle{Certificado de Aprobación}
}{
    \cvtitle{Armador y Reparados de PC}
    Curso Presencial Dictado y Certificado por Asociación Trabajadores del Estado (ATE), Reconocido por el Ministerio de Educación de la Provincia de Santa Fe. Según disposición Nro. 25/09.
    Duración: 60 hs. cátedra.
}

% skills
% ------

\cvsection{IDIOMAS}

\vspace{\cvbetweensectionandheadingextraskipamount}

% languages
\cvitem{
    \cvheadingstyle{Español}
}{
    Nativo
}

\cvitem{
    \cvheadingstyle{Inglés}
}{
    Elemental
}

\newpage
% Habilidades Informáticas
\cvsection{HABILIDADES INFORMÁTICAS}

\cvitem{
    \cvheadingstyle{Conocimiento Básico}
}{
    NodeJs, ExpressJs, .Net, Prolog, Scheme
}

\cvitem{
    \cvheadingstyle{Conocimiento Intermedio}
}{
     Php, Html, Css, Javascript, JQuery, Laravel, VueJS, MySql, Postgre SQL, Java,  C++, Sql Server, Git, GitHub, LaTeX, Wordpress, MatLab, OpenCv, Joomla,  JasperReports, Gnu/linux, Windows,  Excel, Word, Power-Point.
}

\cvitem{
    \cvheadingstyle{Hardware}
}{
    Mantenimiento y Reparación de PC.
}

% additional info
% ---------------

\cvsection{INFORMACIÓN ADICIONAL}

\vspace{\cvbetweensectionandheadingextraskipamount}

% driving licence
\cvitem{
    \cvheadingstyle{Licencia de Conducir}
}{
    Disponibilidad para viajar.
}

% interests
\cvitem{
    \cvheadingstyle{Intereses}
}{
    Tecnología, Open-Source, Programación, Big Data, IoT, Business Intelligence, Inteligencia Computacional, Procesamiento Digital de Imágenes, Metodologías de Desarrollo Ágil de Software
}

% actividades
\cvitem{
    \cvheadingstyle{Actividades}
}{
    Gimnasio, Correr, Deportes especialmente rugby
}

% otros datos
\cvitem{
    \cvheadingstyle{Otros datos}
}{
    \begin{itemize}
        \item Actitud Proactiva
        \item Rápida asimilación de Contenidos
        %\item Predisposición para el trabajo en equipo
        \item Habilidades comunicativas con predisposición para el trabajo en equipo
    \end{itemize}
}

\end{document}